%! Author = Nectarios Chroniaris
%! Date = 2022-02-28

% Preamble
\documentclass[11pt]{article}
\usepackage[utf8]{inputenc}
\PassOptionsToPackage{hyphens}{url}\usepackage{hyperref}
\usepackage{listings}
\usepackage{verbatim}
\usepackage{makecell}
\usepackage{xcolor}
\usepackage{amsmath}

\newcommand{\ts}{\textsuperscript}

\definecolor{verylightgray}{gray}{0.95}

% https://tex.stackexchange.com/a/119864
\lstset{
    frame=single,
    framesep=6pt,
    framerule=0pt,
    xleftmargin=8pt,
    mathescape=false,
    backgroundcolor=\color{verylightgray},
    aboveskip=3mm,
    belowskip=3mm,
    showstringspaces=false,
    columns=flexible,
    basicstyle={\small\ttfamily},
    numbers=none,
    numberstyle=\tiny\color{gray},
    keywordstyle=\color{blue},
    commentstyle=\color{dkgreen},
    stringstyle=\color{mauve},
    breaklines=true,
    breakatwhitespace=true,
    tabsize=3
}

%! suppress = LineBreak
%! suppress = TooLargeSection
\begin{document}

    \title{magpieCTF 2022 - Turn-Key Writeup}
    \author{Nectarios Chroniaris}

    \maketitle


    \section{Preamble}\label{sec:preamble}

    The intended solve, as documented on \href{https://github.com/infosec-ucalgary/magpieCTF2022-public/tree/main/challenges/networks/turn-key}{magpieCTF's public GitHub repo}, involves renting/using servers that are physically close to the vaults, in order to get each request to complete in the required 300 or so milliseconds.

    \bigskip

    However, by taking some clever shortcuts in the design of the protocol, we can reduce round-trip delays and otherwise unnecessary overhead, to squeeze the total delay under the required amount (yes, even for the server in Bangalore!).


    \section{Abridged Solution}\label{sec:abridged-solution}

    \begin{enumerate}
        \item Read the specification and understand what messages are required to fulfill the protocol.
        \begin{itemize}
            \item Use \verb`nmap` to scan the vault servers to see which port the protocol is hosted under.
            \item \verb`nmap -p 1-65535 -T4 -A -v vault1.momandpopsflags.ca`
            \item i.e.\ ``scan all ports from 1--65535, be aggressive (\verb`-T4`, \verb`-A`) and log verbosely''.
        \end{itemize}
        \item After finding the correct port (thankfully they are all the same!), try to connect with \verb`nc` to see if it works.
        \begin{itemize}
            \item \verb`nc vault1.momandpopsflags.ca 5555`
            \item Test out the protocol, realize that your human fingers are much too slow to get the partial flag.
        \end{itemize}
        \item Start to write a script in your programming language of choice that can do TCP socket programming.
        \begin{itemize}
            \item The easiest IMO is Python 3, using the \verb`socket` library.
            \item Don't worry about efficiency in this step, just make sure that it works (the protocol successfully terminates).
            \item If you can get one vault, great! Remember we have to get all 3 at the same time.
        \end{itemize}
        \item Find room to improve efficiency and reduce overhead.
        \begin{itemize}
            \item Realize that you don't have to waste time reading, writing, and reading, and writing before something useful happens. You can just \textbf{immediately} send BOTH messages since the first two responses from the server are neither useful nor have any dynamic information.
            \item Remember it's fine to hardcode stuff here, because we want to break this specific protocol, and not have a general solution!
        \end{itemize}
        \item Run against all the servers at the same time to get all the necessary information for the key.
        \begin{itemize}
            \item Assuming your script is fast enough, you can run it all from the same machine using shell trickery:
            \item \verb`python3 <script> vault1 &` \\
            \verb`    python3 <script> vault2 &` \\
            \verb`    python3 <script> vault3 & wait`
        \end{itemize}
        \item Decrypt the flag using the 3-part key, initial value, and ciphertext from the vaults, using any tool of your choice.
        \begin{itemize}
            \item \href{https://gchq.github.io/CyberChef}{CyberChef is pretty convenient}, but any tool works, as there is nothing special going on here besides the key being split up into 3 pieces.
        \end{itemize}
    \end{enumerate}

    \pagebreak


    \section{Full Solution \& Thought Process}\label{sec:full-solution-thought-process}

    \subsection{Specification}\label{subsec:specification}

    Every good protocol starts off with a specification. We want to read this thoroughly and make sure we understand it, because then scripting will be easier. Here we note that:

    \begin{itemize}
        \item There are 3 vaults at different addresses.
        \item The protocol does not seem to rely on any other application-level protocol (like HTTP, for example), so it's likely just raw TCP messaging.
        \item The protocol involves us sending strings back to one another.
        \item In the final step we have to send a challenge string back.
        \begin{itemize}
            \item This means we have to make sure we read the data the server sent us so we can send it back.
        \end{itemize}
        \item There is no mention of what port(s) this service runs on, so we will have to find this ourselves.
        \item We have to get responses from all servers within 2.4s or else the encryption rotates.
        \item We know what output to expect in both the successful and unsuccessful cases.
    \end{itemize}

    \subsection{Finding Ports}\label{subsec:finding-ports}

    Fresh off our exciting specification adventure, it seems that the first step is to find where the service we want is hosted. We know the IP addresses (\verb`vaultX.momandpopsflags.ca`), but not the port.

    Pick one of the three servers to scan, and use any port scanning tool of your choice. I chose to use \verb`nmap`, since it was installed on my system.

    \begin{lstlisting}[gobble=8,label={lst:nmap-command}]
        $ nmap -p 1-65535 -T4 -A -v vault1.momandpopsflags.ca
    \end{lstlisting}

    \pagebreak

    \noindent A quick explanation of the options:

    \begin{table*}[h!]
        \centering
        \label{tab:nmap-command}
        \begin{tabular}{|r|l|}
            \hline
            \textbf{Flag} & \textbf{Description}                            \\ \hline
            \verb`-p N-M` & scan ports from \verb`N` to \verb`M`            \\ \hline
            \verb`-T4`    & Aggressive timing template, speeds up execution \\ \hline
            \verb`-A`     & Aggressive scan, gives more information         \\ \hline
            \verb`-v`     & Verbose logging                                 \\ \hline
        \end{tabular}
    \end{table*}

    After running this (it might take a couple of minutes!) you should see an open TCP port at 5555. That looks pretty out-of-place, so that's probably the one we want. In the next section we'll verify this by using netcat (\verb`nc`) to poke the service and see if it's the right one.

    \subsection{Playing Around With netcat}\label{subsec:playing-around-with-nc}

    \subsection{Naive Script}\label{subsec:naive-script}

    \subsection{Improved Script}\label{subsec:improved-script}

    \subsection{Turn All The Keys!}\label{subsec:turn-all-the-keys}

    \subsection{Decrypting the Flag}\label{subsec:decrypting-the-flag}

    % (1/3): 0x9DEFE0C566FE0264664BE
    % IV: 0x4F5C25C4954F273472EC67AC494DADBD
    % Ciphertext: 0x917774AA5412A434FA9EC619585D07E8DF1A48952E15ACCFD702F7B47F2C8610685164ACF4C4CE919C4F436615CFD275

    % (2/3): 0x44D8BFCD0F625A8182A13
    % IV: 0x4F5C25C4954F273472EC67AC494DADBD
    % Ciphertext: 0x917774AA5412A434FA9EC619585D07E8DF1A48952E15ACCFD702F7B47F2C8610685164ACF4C4CE919C4F436615CFD275

    % (3/3): 0xFEE73E2A20D92242FA45B5
    % IV: 0x4F5C25C4954F273472EC67AC494DADBD
    % Ciphertext: 0x917774AA5412A434FA9EC619585D07E8DF1A48952E15ACCFD702F7B47F2C8610685164ACF4C4CE919C4F436615CFD275


    \section{Conclusion}\label{sec:conclusion}


\end{document}
